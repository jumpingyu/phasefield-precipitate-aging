\documentclass[10pt]{article}
\usepackage{amsmath,amssymb,graphicx,mathtools,siunitx}
\usepackage[letterpaper,top=1.0in,bottom=1.0in,left=1.0in,right=1.0in]{geometry}
\usepackage[version=3]{mhchem}

\DeclareMathOperator{\sgn}{sgn}

\pagestyle{plain}
\begin{document}
	I am attempting to model solid-state transformations in Inconel 625 based on a published Inconel 718 model \cite{Zhou2014},
	which is a generalization of the KKS binary model \cite{Kim1999}.
	Model parameters are listed in Appendix~\ref{app:params}.
	
	To capture $\delta$ and Laves precipitates in a $\gamma$ matrix, I have chosen Ni--\SI{30}{\percent} Cr--\SI{2}{\percent} Nb as the model system.
	The interdendritic regions in additive manufacturing get enriched to Ni--\SI{31}{\percent} Cr--\SI{13}{\percent} Nb.
	The four-phase three-component model is represented using two composition fields $\left(x_{\ce{Cr}}, x_{\ce{Nb}}\right)$
	and two phase fields $\left(\phi_{\delta}, \phi_{\mathrm{Laves}}\right)$.
	The CALPHAD database was modified from Du \emph{et al.} \cite{Du2005} to make $\delta$ a line compound.


	\section{Phase Field Model}
		In this model, the system composition depends on the pure-phase compositions and phase fractions:
		\begin{align}
			x_{\ce{Cr}} &= \left( 1-\sum h(|\phi_i|) \right) x_{\ce{Cr}}^{\gamma}
			             + h(|\phi_{\delta}|) x_{\ce{Cr}}^{\delta}
			             + h(|\phi_{\mathrm{L}}|) x_{\ce{Cr}}^{\mathrm{L}}
						\label{eqn:conscr}\\
			x_{\ce{Nb}} &= \left( 1-\sum h(|\phi_i|) \right) x_{\ce{Nb}}^{\gamma}
			             + h(|\phi_{\delta}|) x_{\ce{Nb}}^{\delta}
			             + h(|\phi_{\mathrm{L}}|) x_{\ce{Nb}}^{\mathrm{L}}
 			             \label{eqn:consnb}\\
 			h(\phi)     &= \phi^3\left(6\phi^2 - 15\phi + 10\right)\label{eqn:interp}
		\end{align}
		Therefore, the ternary model is implemented with two ``real'' compositions $(x_{\ce{Cr}}, x_{\ce{Nb}})$ and
		six `fictitious'' compositions $(x_{\ce{Cr}}^{\gamma}, x_{\ce{Nb}}^{\gamma}, x_{\ce{Cr}}^{\delta}, x_{\ce{Nb}}^{\delta}, x_{\ce{Cr}}^{\mathrm{L}}, x_{\ce{Nb}}^{\mathrm{L}})$.

		Zhou \emph{et al.} \cite{Zhou2014} defines the order parameter such that $\phi_i=\pm1$ indicates presence of the phase $i$, and $\phi_i=0$ indicates absence.
		The matrix phase $\gamma$ exists where $\sum h(|\phi_i|)=0$. This allows for multiple discrete precipitates of the same phase, without unphysical coalescence.
		The free energy density is
		\begin{align}
		    \nonumber
			f(x,\phi,t) &= \left( 1-\sum h(|\phi_i|) \right) f_{\gamma}(x_{\ce{Cr}}^{\gamma},x_{\ce{Nb}}^{\gamma})
			             + h(|\phi_{\delta}|) f_{\delta}(x_{\ce{Cr}}^{\delta},x_{\ce{Nb}}^{\delta})
			   		     + h(|\phi_{\ce{L}}|)f_{\ce{L}}(x_{\ce{Nb}}^{\ce{L}},x_{\ce{Ni}}^{\ce{L}})\\
		    \nonumber
			            &+ \omega_{\delta}(\phi_{\delta})^2(1-|\phi_{\delta}|)^2
			             + \omega_{\ce{L}}(\phi_{\ce{L}})^2(1-|\phi_{\ce{L}}|)^2\\
			            &+ \alpha\phi_{\delta}^2\phi_{\ce{L}}^2
			            \label{eqn:density}
		\end{align}
		with the elastic energy of the Zhou model neglected here.
		The first line weighs the single-phase free energy expressions by their respective phase fractions,
		the second establishes double-well potentials between matrix ($\gamma$, $\phi=0$) and the precipitate phases ($\phi=\pm1)$,
		and the third line penalizes triple junctions and coalescence of precipitate phases.

		The KKS interface model \cite{Kim1999} assumes constant chemical potential through the interface, so
		\begin{align}
			\label{eqn:potcr}
			\tilde{\mu}_{\ce{Cr}} &= \frac{\partial f_{\gamma}}{\partial x_{\ce{Cr}}^{\gamma}}
			                       = \frac{\partial f_{\delta}}{\partial x_{\ce{Cr}}^{\delta}}
			                       = \frac{\partial f_{\ce{L}}}{\partial x_{\ce{Cr}}^{\ce{L}}}\\
			\label{eqn:potnb}
			\tilde{\mu}_{\ce{Nb}} &= \frac{\partial f_{\gamma}}{\partial x_{\ce{Nb}}^{\gamma}}
			                       = \frac{\partial f_{\delta}}{\partial x_{\ce{Nb}}^{\delta}}
			                       = \frac{\partial f_{\ce{L}}}{\partial x_{\ce{Nb}}^{\ce{L}}}
		\end{align}
		The pure phase compositions $\left(x_j^i\right)$ are determined by solving the parallel tangent construction constrained by the
		conservation of mass

		\begin{align}
			\label{eqn:Jf1}
			0 &= x_{\ce{Cr}} - \left( 1-\sum h(|\phi_i|) \right)x_{\ce{Cr}}^{\gamma}
			                 - h(|\phi_{\delta}|)x_{\ce{Cr}}^{\delta}
			                 - h(|\phi_{\ce{L}}|)x_{\ce{Cr}}^{\ce{L}}\\
			\label{eqn:Jf2}
			0 &= x_{\ce{Nb}} - \left( 1-\sum h(|\phi_i|) \right)x_{\ce{Nb}}^{\gamma}
			                 - h(|\phi_{\delta}|)x_{\ce{Nb}}^{\delta}
			                 - h(|\phi_{\ce{L}}|)x_{\ce{Nb}}^{\ce{L}}
		\end{align}\\
		and equality of chemical potentials for each phase,\\
		\begin{minipage}{0.45\textwidth}
		\begin{align}
			\label{eqn:Jf3}
			0 &= \frac{\partial f_{\gamma}}{\partial x_{\ce{Cr}}^{\gamma}} - \frac{\partial f_{\delta}}{\partial x_{\ce{Cr}}^{\delta}}\\
			\label{eqn:Jf4}
			0 &= \frac{\partial f_{\gamma}}{\partial x_{\ce{Nb}}^{\gamma}} - \frac{\partial f_{\delta}}{\partial x_{\ce{Nb}}^{\delta}}
		\end{align}
		\end{minipage}\qquad
		\begin{minipage}{0.45\textwidth}
		\begin{align}
			\label{eqn:Jf7}
			0 &= \frac{\partial f_{\gamma}}{\partial x_{\ce{Cr}}^{\gamma}} - \frac{\partial f_{\ce{L}}}{\partial x_{\ce{Cr}}^{\ce{L}}}\\
			\label{eqn:Jf8}
			0 &= \frac{\partial f_{\gamma}}{\partial x_{\ce{Nb}}^{\gamma}} - \frac{\partial f_{\ce{L}}}{\partial x_{\ce{Nb}}^{\ce{L}}}
		\end{align}
		\end{minipage}\\
		in which each partial derivative is evaluated at the pure phase composition $x_j^i$, not the system composition $x_j$.
		This set of eight equations should uniquely solve for the eight unknown pure compositions at each point,
		given that $x_{\ce{Ni}}^i = 1-x_{\ce{Cr}}^i-x_{\ce{Nb}}^i$.
		This solution is found using the GNU Scientific Library's multiroot solver, provided these eight equations and the
		Jacobian matrix defined by their partial derivatives with respect to $x_{\ce{Cr}}^{\gamma}$,
		                                                                     $x_{\ce{Nb}}^{\gamma}$,
		                                                                     $x_{\ce{Cr}}^{\delta}$,
		                                                                     $x_{\ce{Nb}}^{\delta}$,
		                                                                     $x_{\ce{Cr}}^{\ce{L}}$, and
		                                                                     $x_{\ce{Nb}}^{\ce{L}}$.
		The complete Jacobian matrix is written in Appendix~\ref{app:jacobian}.

		The first guess for each composition is made using line compound approximations for each phase,
		applied after the initial condition $(t=0)$.
		Subsequent guesses simply copy the result of the previous timestep.
		During the iterations, $x_{\ce{Cr}}$,
		                       $x_{\ce{Nb}}$,
		                       $\phi_{\delta}$, and
		                       $\phi_{\ce{L}}$ are held constant.
		In the event that a valid solution cannot be found, at any time during the simulation,
		the line compound guesses are applied.
		Details are provided in Appendix~\ref{app:linecomp}.


	\section{Thermodynamic Model}
		The pure phase free energies depend on Gibbs free energy expressions, divided by molar volume to convert from \si{\joule/\mole} to \si{\joule/\cubic\meter}.
		The Gibbs free energy expressions are read from a CALPHAD database (Du \emph{et al.} \cite{Du2005}) using pycalphad.
		This substitution is not valid over the entire ternary composition space, and sometimes incurs kinks.

		CALPHAD represents phases in terms of sublattice compositions $y$, not system compositions $x$.
		This is fundamental to CALPHAD.
		Each sublattice has a site fraction $a$, similar to the subscripts in a molecular formula.
		%For any phase with components $j$, sublattices $k$, and site fractions $a$,
		%the sum of the sublattice compositions for each element $\sum y_j^k=1$.
		The sublattice compositions are constrained by mass conservation:
		\begin{align}
			a' + a'' + \cdots &= 1\\\nonumber
			a'y_A' + a''y_A'' + \cdots &= x_A\\\label{eqn:sublattice}
			a'y_B' + a''y_B'' + \cdots &= x_B
		\end{align}
		where $'$ indicates the first sublattice, $''$ the second, etc.
		The lattice fractions are also conserved:
		\begin{align}\nonumber
			y_A' + y_B' + \cdots &= 1\\
			y_A'' + y_B'' + \cdots &= 1
		\end{align}
		These constraints can, for some phases, be used to map sublattice compositions directly into system compositions.
		If all but one component appear on only one sublattice each, then $y$ maps into $x$ uniquely.

		\begin{itemize}
		\item $\gamma$ is represented with one sublattice \cite{Du2005}, \ce{(Cr{,}Nb{,}Ni)1}.
			This maps trivially:
			\begin{align*}
				y_{\ce{Cr}}' &= x_{\ce{Cr}} & & & &\\
				y_{\ce{Nb}}' &= x_{\ce{Nb}} & & & &\\
				y_{\ce{Ni}}' &= x_{\ce{Ni}} & & & &
			\end{align*}
		
		\item $\delta$ is represented with two sublattices \cite{Du2005}, \ce{(Nb{,}Ni)_{1/4}(Cr{,}Nb{,}Ni)_{3/4}}.
			Since \ce{Nb} and \ce{Ni} appear on both, this cannot be mapped.
			However, if we assume that \ce{Nb} partitions to the first sublattice,
			\ce{(Nb{,}Ni)_{1/4}(Cr{,}Ni)_{3/4}} can be solved.
			\begin{align*}
				             &                   & y_{\ce{Cr}}'' &= \frac{4}{3}x_{\ce{Cr}}     & x_{\ce{Cr}}&<\frac{3}{4}\\
				y_{\ce{Nb}}' &= 4x_{\ce{Nb}}     &               &                             & x_{\ce{Nb}}&<\frac{1}{4}\\
				y_{\ce{Ni}}' &= 1 - 4x_{\ce{Nb}} & y_{\ce{Ni}}'' &= 1 - \frac{4}{3}x_{\ce{Cr}} &            &\\
			\end{align*}
		
		\item Laves is represented with two sublattices, \ce{(Cr{,}Nb{,}Ni)_{2/3}(Cr{,}Nb)_{1/3}}.
			Since \ce{Cr} and \ce{Nb} appear on both sublattices, this cannot be mapped.
			Again, assuming that \ce{Nb} partitions to the second lattice,
			\ce{(Cr{,}Ni)_{2/3}(Cr{,}Nb)_{1/3}} can be solved.
			\begin{align*}
				y_{\ce{Cr}}' &= 1 - \frac{3}{2}x_{\ce{Ni}} & y_{\ce{Cr}}'' &= 1 - 3x_{\ce{Nb}} &             &\\
				             &                             & y_{\ce{Nb}}'' &= 3x_{\ce{Nb}}     & x_{\ce{Nb}} &< \frac{1}{3}\\
				y_{\ce{Ni}}' &= \frac{3}{2}x_{\ce{Ni}}     &               &                   & x_{\ce{Ni}} &< \frac{2}{3}\\
			\end{align*}
		\end{itemize}		

		\subsection{Taylor series approximation}
		Since smooth and continuously differentiable functions are needed, the single-phase free energies are approximated
		by second-order Taylor series expansion about control compositions $\mathbf{a}=(a_{\ce{Cr}}, a_{\ce{Nb}}, a_{\ce{Ni}})$
		chosen as the vertices of the triangular phase field of $\gamma$--$\delta$--\ce{Laves} coexistence predicted by the CALPHAD phase diagram.
		For each phase $\alpha\in(\gamma,\delta,\ce{Laves})$,
		\begin{align}\nonumber
			f_{\alpha} &\approx f_{\alpha}\left(a_{\ce{Cr}}^{\alpha}, a_{\ce{Nb}}^{\alpha}\right)\\\nonumber
	                            &+ \sum\limits_{i}\frac{\partial f_{\alpha}\left(a_{\ce{Cr}}^{\alpha}, a_{\ce{Nb}}^{\alpha}\right)}{\partial x_i^{\alpha}}
	                              \left(x_i^{\alpha} - a_i^{\alpha}\right)\\
	                            &+ \sum\limits_{i}\sum\limits_{j}\frac{1}{2}\frac{\partial^2 f_{\alpha}\left(a_{\ce{Cr}}^{\alpha}, a_{\ce{Nb}}^{\alpha}\right)}
	                              {\partial x_i^{\alpha} \partial x_j^{\alpha}}
	                              \left(x_i^{\alpha} - a_i^{\alpha}\right)\left(x_j^{\alpha} - a_j^{\alpha}\right)
		\end{align}
		with $i,j\in\ce{Cr},\ce{Nb},\ce{Ni}$.
		This is also known as a \emph{paraboloid} free energy approximation.
		Note that, depending on the CALPHAD expression, some terms are null for each phase.


	\section{Equations of Motion for Phases $\{\phi\}$}
		The $\{\phi_i\}$ are not conserved, so Allen-Cahn dynamics are assumed:
		\begin{equation}
			\frac{\partial \phi_i}{\partial t} = -L_i\frac{\delta\mathcal{F}}{\delta\phi_i} = -L_i\left(\frac{\partial f}{\partial \phi_i} - \kappa_i\nabla^2\phi_i\right).
		\end{equation}
		From Eqn.~\ref{eqn:density},
		\begin{align}\nonumber
			\frac{\partial f}{\partial \phi_n} &= -\sgn(\phi_i)h'(|\phi_n|)\left[f_{\gamma}(x_{\ce{Cr}}^{\gamma},x_{\ce{Nb}}^{\gamma}) - f_n(x_{\ce{Cr}}^n,x_{\ce{Nb}}^n)\right]
			                                    + \left[1 - \sum h(|\phi_i|)\right] \frac{\partial f_{\gamma}}{\partial \phi_n}
			                                    + h(|\phi_n|)\frac{\partial f_n}{\partial \phi_n}\\
			                                   &+ 2\omega_n\phi_n\left(1-|\phi_n|\right)^2 - 2\omega_n\phi_n^2\sgn(\phi_n)\left(1-|\phi_n|\right)
			                                    + 2\alpha\phi_n\sum_{i\neq n}\phi_i^2.
		\end{align}
		Invoking the multivariable chain rule and chemical potential (Eqns.~\ref{eqn:potcr}~and~\ref{eqn:potnb}),
		\begin{align}\nonumber
			\frac{\partial f_{\alpha}}{\partial \phi_{\alpha}} &= \sum_j \frac{\partial f_{\alpha}}{\partial x_j^{\alpha}}\frac{\partial x_j^{\alpha}}{\partial \phi_{\alpha}}
			                                         = \sum_j\frac{\partial x_j^{\alpha}}{\partial \phi_{\alpha}}\tilde{\mu}_j\\
			\nonumber
			\frac{\partial f}{\partial \phi_n} &= -\sgn(\phi_n)h'(|\phi_n|)\left[f_{\gamma}(x_{\ce{Cr}}^{\gamma},x_{\ce{Nb}}^{\gamma}) - f_n(x_{\ce{Cr}}^n,x_{\ce{Nb}}^n)\right]
			                                    + \sum_j\left(\left[1 - \sum h(|\phi_i|)\right] \frac{\partial x_j^{\gamma}}{\partial \phi_n}
			                                    + h(|\phi_n|)\frac{\partial x_j^n}{\partial \phi_n}\right)\tilde{\mu}_j\\
			                                   &+ 2\omega_n\phi_n\left(1-|\phi_n|\right)\left[1 - h(|\phi_n|) - \sgn(\phi_n)\phi_n\right]
			                                    + 2\alpha\phi_n\sum_{i\neq n}\phi_i^2.\label{eqn:potentphi}
		\end{align}

		Implicitly differentiating both sides of the expression for system composition,
		Eqns.~\ref{eqn:conscr} and~\ref{eqn:consnb}, with respect to a phase,\footnote{
		Cf. Eqn.~6.91 in Provatas and Elder \cite{Provatas2010}. The amount of species should not change explicitly with changes in phase.
		}
		\begin{align}
			\frac{\partial x_j}{\partial \phi_n} &= -\sgn(\phi_n)h'(|\phi_n|)\left[x_j^{\gamma} - x_j^n\right]
			                                          + \left[1 - \sum h(|\phi_i|)\right] \frac{\partial x_j^{\gamma}}{\partial \phi_n}
			                                          + h(|\phi_n|) \frac{\partial x_j^{n}}{\partial \phi_n}\\
			\frac{\partial x_j}{\partial \phi_n} &\equiv 0\\
			\sgn(\phi_n)h'(|\phi_n|)\left[x_j^{\gamma} - x_j^n\right] &= \left[1 - \sum h(|\phi_i|)\right] \frac{\partial x_j^{\gamma}}{\partial \phi_n}
			                                                           + h(|\phi_n|) \frac{\partial x_j^{n}}{\partial \phi_n}.
			                                                           \label{eqn:xdiff}
		\end{align}
		Substituting Eqn.~\ref{eqn:xdiff} into Eqn.~\ref{eqn:potentphi} and simplifying,
		we arrive at the final result:
		\begin{align}\nonumber
			\frac{\partial f}{\partial \phi_n} &= -\sgn(\phi_n)h'(|\phi_n|)\left(f_{\gamma}(x_{\ce{Cr}}^{\gamma},x_{\ce{Nb}}^{\gamma})
			                                    - f_n(x_{\ce{Cr}}^n,x_{\ce{Nb}}^n) - \sum_j\left[x_j^{\gamma} - x_j^n\right]\tilde{\mu}_j\right)\\
			                                   &+ 2\omega_n\phi_n\left(1-|\phi_n|\right)\left[1 - h(|\phi_n|) - \sgn(\phi_n)\phi_n\right]
			                                    + 2\alpha\phi_n\sum_{i\neq n}\phi_i^2.
			\label{eqn:phieom}
		\end{align}		


	\section{Equations of Motion for Compositions $\{x\}$}
		Composition is conserved, so we choose Cahn-Hilliard dynamics for $x$:
		\begin{equation}
			\frac{\partial x_{\ell}}{\partial t} = \nabla\cdot\sum_k M_{\ell k}\nabla\frac{\delta\mathcal{F}}{\delta x_k}
			                                     = \nabla\cdot\sum_k M_{\ell k}\nabla\frac{\partial f}{\partial x_k}.\label{eqn:fick}
		\end{equation}
		
		Differentiating Eqn.~\ref{eqn:density}, then applying the chain rule and the definition of chemical potential,
		\begin{align}
			\frac{\partial f}{\partial x_k} &= \left[1-\sum_{\alpha}h(|\phi_{\alpha}|)\right]\frac{\partial f_{\gamma}}{\partial x_k}
			                                 + \sum_{\alpha}h(|\phi_{\alpha}|)\frac{\partial f_{\alpha}}{\partial x_k}\\
			                                &= \sum_j\left[1-\sum_{\alpha}h(|\phi_{\alpha}|)\right]\frac{\partial f_{\gamma}}{\partial x_j^{\gamma}}\frac{\partial x_j^{\gamma}}{\partial x_k}
			                                 + \sum_j\sum_{\alpha}h(|\phi_{\alpha}|)\frac{\partial f_{\alpha}}{\partial x_j^{\alpha}}\frac{\partial x_j^{\alpha}}{\partial x_k}\\
			                                &= \sum_j\left(\left[1-\sum_{\alpha}h(|\phi_{\alpha}|)\right]\frac{\partial x_j^{\gamma}}{\partial x_k}\mu_j^{\gamma}
			                                 + \sum_{\alpha}h(|\phi_{\alpha}|)\frac{\partial x_j^{\alpha}}{\partial x_k}\mu_j^{\alpha}\right).
		\end{align}
		Since the pure-phase composition $x_j^{\alpha}$ depends only on $x_j$, $\frac{\partial x_j^{\alpha}}{\partial x_k} = \delta_{jk}$, and
		\begin{equation}
			\frac{\partial f}{\partial x_k} = \left[1-\sum_{\alpha}h(|\phi_{\alpha}|)\right]\mu_k^{\gamma} + \sum_{\alpha}h(|\phi_{\alpha}|)\mu_k^{\alpha}.
		\end{equation}
		In this phase field formulation, $\tilde{\mu}_k\equiv\mu_k^{\gamma}=\mu_k^{\alpha}$ (Eqn.~\ref{eqn:potcr}) and
		\begin{equation}
			\frac{\partial f}{\partial x_k} = \tilde{\mu}_k.\label{eqn:potentc}
		\end{equation}
		Substituting Eqn.~\ref{eqn:potentc} into Eqn.~\ref{eqn:fick}, with a proportionality constant $V_m^2$, we 
		\begin{equation}
			\frac{\partial x_{\ell}}{\partial t} = V_m^2 \nabla\cdot\sum_k M_{\ell k}\nabla\tilde{\mu}_k.
		\end{equation}
		Since we do not have detailed interfacial data, the mobility matrix is diagonal and $M_{\ell k} = \delta_{\ell k}M_{\ell}$,
		\emph{i.e.} the mobility of element $\ell$ only depends on its own concentration field.
		Using this simplification, we can recover the form presented by Zhou \emph{et al.} \cite{Zhou2014}:
		\begin{equation}
			\frac{\partial x_{\ell}}{\partial t} = V_m^2 \nabla \cdot M_{\ell} \nabla \tilde{\mu}_k.\label{eqn:zhoudiff}
		\end{equation}

		If we further invoke the chain rule,
		\begin{align}
			\frac{\partial x_{\ell}}{\partial t} &= V_m^2 \nabla\cdot M_{\ell}\sum_j\frac{\partial \tilde{\mu}_k}{\partial x_j^{\gamma}}\nabla x_j^{\gamma}\\
			                                     &= V_m^2 \nabla\cdot M_{\ell}\sum_j\frac{\partial^2 f_{\gamma}}{\partial x_{\ell}^{\gamma} \partial x_j^{\gamma}}\nabla x_j^{\gamma}
		\end{align}
		For the specific case of paraboloid approximations to the pure free energy expressions, the cross-terms of curvature are zero,
		and we have an equivalent result in terms of composition rather than chemical potential:
		\begin{equation}
			\frac{\partial x_{\ell}}{\partial t} = 2V_m^2 \nabla\cdot M_{\ell}C_{\ell}^{\gamma}\nabla x_{\ell}^{\gamma},\label{eqn:interdiff}
		\end{equation}
		with curvature $C_{\ell}^{\gamma}=\frac{1}{2}\frac{\partial^2 f_{\gamma}}{\partial \left({x_{\ell}^{\gamma}}\right)^2}$.
		
		We can also use the definition of diffusivity:
		\begin{align}
			D_{\ell j}^{\alpha} &= \sum_k M_{\ell k}^{\alpha}\frac{\partial^2 f_{\gamma}}{\partial x_{\ell}^{\alpha} \partial x_j^{\alpha}}\\
			\frac{\partial x_{\ell}}{\partial t} &= \nabla\cdot\sum_j D_{\ell j}^{\gamma} \nabla x_j^{\gamma}.
		\end{align}

		For Cr--Nb--Ni, we have diffusivity data at \SI{1273}{\kelvin} \cite{Xu2016};
		for a matrix of Ni--\SI{26.4}{\percent} Cr--\SI{1.3}{\percent} Nb,
		\begin{equation}\label{eqn:diffusion}
			\tilde{D}^{\ce{Ni}} = \left[\begin{array}{cc}
			                            \tilde{D}_{\mathrm{CrCr}} & \tilde{D}_{\mathrm{CrNb}}\\
			                            \tilde{D}_{\mathrm{NbCr}} & \tilde{D}_{\mathrm{NbNb}}
		                                \end{array}
		                          \right]
			                    = \left[\begin{array}{cc}
			                            2.16 & 2.97\\
			                            0.56 & 4.29
		                                \end{array}
		                          \right] \times \SI{e-15}{\meter\squared\per\second}.
		\end{equation}
		If we treat diffusivity as temperature-independent, then
		\begin{equation}
			\frac{\partial x_{\ell}}{\partial t} = D_{\ell\ce{Cr}}^{\gamma}\nabla^2 x_{\ce{Cr}}^{\gamma} + D_{\ell\ce{Nb}}^{\gamma}\nabla^2 x_{\ce{Nb}}^{\gamma}.
		\end{equation}
	

	\section{Timestep Adaptivity}
		If the phase-field $\phi$ advances more than one mesh point in a given timestep, the solution is unstable.
		In practice, $\phi$ should advance $0.1\Delta x$ or less.
		Using the advection equation,
		\begin{equation}
			\frac{\Delta \phi}{\Delta t} = u|\nabla\phi|
		\end{equation}
		with velocity $u$ normal to the interface,
		we can take a step $\Delta t << \frac{(\Delta x)^2}{4D}$
		and solve for
		\begin{equation}
			\Delta t^* = \frac{0.1\Delta x}{u},
		\end{equation}
		the timestep we ``should'' have taken to move the interface as quickly as possible.
		If the global maximum $\Delta t^*$ is smaller than the actual $\Delta t$, we can speed up for the next step.
		Otherwise, if $\Delta t > \Delta t^*$, the current step is invalid and must be repeated with a smaller timestep.
		Useful scaling factors were empirically found to be $1.00001 \Delta t$ to accelerate and $0.9 \Delta t$ to brake,
		although braking is best avoided.
		


	\appendix
	\newpage
	\section{Model parameters}\label{app:params}
		\begin{table}[ht]\centering
			\caption{Model parameters used in this work}
			\begin{tabular}{lll}\hline
				Parameter            & Symbol                   & Value\\\hline
				Mesh resolution      & $\Delta x$               & \SI{5.0e-9}{\meter}\\
				Timestep             & $\Delta t$               & \SI{5.0e-7}{\second}\\
				Temperature          & $T$                      & \SI{870}{\degreeCelsius}\\
				Molar volume         & $V_m$                    & \SI{1.0e-5}{\cubic\meter/\mole}\\
				Trijunction penalty  & $\alpha$                 & \SI{1.07e11}{\joule/\cubic\meter}\\
				Interfacial energy   & $\sigma_{\delta}
				                       =\sigma_{\ce{L}}$          & \SI{1.01}{\joule/\square\meter}\\
				Gradient penalty     & $\kappa_{\delta}
				                       =\kappa_{\ce{L}}$          & \SI{1.24e-8}{\joule/\meter}\\
				Mobility             & $M_{\ce{Cr}}
				                       = M_{\ce{Nb}}$           & \SI{2.42e-18}{\square\mole/\newton\second\square\meter}\\
				Diffusivity          & $D_{\ce{Cr}}
				                       = D_{\ce{Nb}}$           & \SI{1.58e-16}{\square\meter/\second}\\
				Mobility             & $L_{\delta}
				                       =L_{\ce{L}}$               & \SI{2.904e-11}{\square\meter/\newton/\second}\\
				Interface width      & $2\lambda$               & $7\Delta x$\\
				Interface width      & $2\lambda$               & \SI{35e-9}{\meter}\\
				Well height          & $\omega_{\delta}
				                       =\omega_{\ce{L}}$          & $6.6 \sigma_{\delta} / 2\lambda$\\
				Well height          & $\omega_{\delta}
				                       =\omega_{\ce{L}}$          & \SI{1.9e8}{\joule/\cubic\meter}\\
				$\gamma$ curvature   & $C_{\ce{Cr}}^{\gamma}$   & \SI{4.8e10}{\joule/\cubic\meter}\\
				                     & $C_{\ce{Nb}}^{\gamma}$   & \SI{6.1e9}{\joule/\cubic\meter}\\
				$\delta$ curvature   & $C_{\ce{Cr}}^{\delta}$   & \SI{5.4e10}{\joule/\cubic\meter}\\
				                     & $C_{\ce{Nb}}^{\delta}$   & \SI{6.8e11}{\joule/\cubic\meter}\\
				Laves curvature      & $C_{\ce{Nb}}^{\ce{L}}$     & \SI{1.2e11}{\joule/\cubic\meter}\\
				                     & $C_{\ce{Ni}}^{\ce{L}}$     & \SI{1.1e10}{\joule/\cubic\meter}\\
				$\gamma$ composition & $\prescript{e}{}{x}_{\ce{Cr}}^{\gamma}$ & \SI{1.00}{\percent}\\
				                     & $\prescript{e}{}{x}_{\ce{Nb}}^{\gamma}$ & \SI{32.3}{\percent}\\
				$\delta$ composition & $\prescript{e}{}{x}_{\ce{Cr}}^{\delta}$ & \SI{0.88}{\percent}\\
				                     & $\prescript{e}{}{x}_{\ce{Nb}}^{\delta}$ & \SI{24.9}{\percent}\\
				Laves composition    & $\prescript{e}{}{x}_{\ce{Nb}}^{\ce{L}}$ & \SI{30.6}{\percent}\\
				                     & $\prescript{e}{}{x}_{\ce{Ni}}^{\ce{L}}$ & \SI{49.1}{\percent}\\
				$\gamma$ minimum     & $f^0_{\gamma}\left(\prescript{e}{}{x}_{\ce{Cr}}^{\gamma}, \prescript{e}{}{x}_{\ce{Nb}}^{\gamma}\right)$
				                                                & \SI{-7.9722e9}{\joule/\cubic\meter}\\
				$\delta$ minimum     & $f^0_{\delta}\left(\prescript{e}{}{x}_{\ce{Cr}}^{\delta}, \prescript{e}{}{x}_{\ce{Nb}}^{\delta}\right)$
				                                                & \SI{-8.5488e9}{\joule/\cubic\meter}\\
				Laves minimum        & $f^0_{\ce{L}}\left(\prescript{e}{}{x}_{\ce{Nb}}^{\ce{L}}, \prescript{e}{}{x}_{\ce{Ni}}^{\ce{L}}\right)$
				                                                & \SI{-8.0522e9}{\joule/\cubic\meter}\\
				\hline
			\end{tabular}
		\end{table}
	
	\section{Units of the Diffusion Equations}
		The expression $\frac{\partial x}{\partial t}$ should have units of \si{\per\second}.
		For Eqn.~\ref{eqn:diffusion}, this is clearly so:
		\begin{align*}
			\frac{1}{\si{\second}} &= \frac{\si{\square\meter}}{\si{\second}}\frac{1}{\si{\square\meter}}\\
			                       &= \frac{1}{\si{\second}}.
		\end{align*}
		
		The unusual units in Eqn.~\ref{eqn:zhoudiff} make the equivalence just slightly less obvious,
		but it is straightforward nonetheless:
		\begin{align*}
			\frac{1}{\si{\second}} &= \frac{\si{\meter^6}}{\si{\square\mole}}
			                          \frac{\si{\square\mole}}{\si{\newton\second\square\meter}}
			                          \frac{\si{\joule}}{\si{\cubic\meter\square\meter}}
			                       = \frac{\si{\meter^6}}{\si{\square\mole}}
			                          \frac{\si{\square\mole\square\second}}{\si{\kilo\gram\meter\second\square\meter}}
			                          \frac{\si{\kilo\gram\square\meter}}{\si{\cubic\meter\square\meter\square\second}}
			                       = \frac{\si{\kilo\gram\meter\tothe8\square\second\square\mole}}{\si{\kilo\gram\meter\tothe8\cubic\second\square\mole}}\\
			                       &= \frac{1}{\si{\second}}.
		\end{align*}

		Eqns.~\ref{eqn:zhoudiff},~\ref{eqn:interdiff},~and~\ref{eqn:diffusion} are interchangeable, simply adjust the timestep to satisfy
		\[
			\Delta t < \mathrm{min}\left(\frac{\left({\Delta x}\right)^2}{8 V_m^2 M_{\ce{Nb}} C_{\ce{Nb}}^{\gamma}}, \frac{\left({\Delta x}\right)^2}{4 D_{\ce{Nb}}}\right).
		\]

	\section{Common Tangent}\label{app:jacobian}
		The Jacobian matrix $\left(J_{ij}=\frac{\partial f_i}{\partial x_j}\right)$ for this system of eight equations, depending on the eight unknown compositions $\{x_{\ce{Cr}}^i\}, \{x_{\ce{Nb}}^i\}$, is written
		
		\begin{table}[ht]\centering
		\begin{small}
		\setlength\extrarowheight{10pt}
		\begin{tabular}{|c|cccccc|}\hline
		$J_{ij}$  & $\partial x_{\ce{Cr}}^{\gamma}$ & $\partial x_{\ce{Nb}}^{\gamma}$ & $\partial x_{\ce{Cr}}^{\delta}$ & $\partial x_{\ce{Nb}}^{\delta}$
		  & $\partial x_{\ce{Cr}}^{\ce{L}}$ & $\partial x_{\ce{Nb}}^{\ce{L}}$\\\hline
		%
		$\partial (\mathrm{Eqn.}~{\ref{eqn:Jf1}})$ & $-1+\sum h(|\phi_i|)$ & 0 & $-h(|\phi_{\delta}|)$ & 0 & $-h(|\phi_{\ce{L}}|)$ & 0\\
		%
		$\partial (\mathrm{Eqn.}~{\ref{eqn:Jf2}})$ & 0 & $-1+\sum h(|\phi_i|)$ & 0 & $-h(|\phi_{\delta}|)$ & 0 & $-h(|\phi_{\ce{L}}|)$\\
		%
		%
		$\partial (\mathrm{Eqn.}~{\ref{eqn:Jf3}})$ & $\frac{\partial^2 f_{\gamma}}{\partial(x_{\ce{Cr}}^{\gamma})^2}$ & $\frac{\partial^2 f_{\gamma}}{\partial x_{\ce{Cr}}^{\gamma}\partial x_{\ce{Nb}}^{\gamma}}$ 
		& $\frac{\partial^2 f_{\delta}}{\partial(x_{\ce{Cr}}^{\delta})^2}$ & $\frac{\partial^2 f_{\delta}}{\partial x_{\ce{Cr}}^{\delta}\partial x_{\ce{Nb}}^{\delta}}$
		& 0 & 0\\
		%
		$\partial (\mathrm{Eqn.}~{\ref{eqn:Jf4}})$ & $\frac{\partial^2 f_{\gamma}}{\partial x_{\ce{Nb}}^{\gamma}\partial x_{\ce{Cr}}^{\gamma}}$ & $\frac{\partial^2 f_{\gamma}}{\partial(x_{\ce{Nb}}^{\gamma})^2}$
		& $\frac{\partial^2 f_{\delta}}{\partial x_{\ce{Nb}}^{\delta} \partial x_{\ce{Cr}}^{\delta}}$ & $\frac{\partial^2 f_{\delta}}{\partial(x_{\ce{Nb}}^{\delta})^2}$
		& 0 & 0\\
		%
		%
		$\partial (\mathrm{Eqn.}~{\ref{eqn:Jf7}})$ & $\frac{\partial^2 f_{\gamma}}{\partial(x_{\ce{Cr}}^{\gamma})^2}$ & $\frac{\partial^2 f_{\gamma}}{\partial x_{\ce{Cr}}^{\gamma}\partial x_{\ce{Nb}}^{\gamma}}$ 
		& 0 & 0 & $\frac{\partial^2 f_{\ce{L}}}{\partial(x_{\ce{Cr}}^{\ce{L}})^2}$ & $\frac{\partial^2 f_{\ce{L}}}{\partial x_{\ce{Cr}}^{\ce{L}}\partial x_{\ce{Nb}}^{\ce{L}}}$
		\\
		%
		$\partial (\mathrm{Eqn.}~{\ref{eqn:Jf8}})$ & $\frac{\partial^2 f_{\gamma}}{\partial x_{\ce{Nb}}^{\gamma}\partial x_{\ce{Cr}}^{\gamma}}$ & $\frac{\partial^2 f_{\gamma}}{\partial(x_{\ce{Nb}}^{\gamma})^2}$
		& 0 & 0 & $\frac{\partial^2 f_{\ce{L}}}{\partial x_{\ce{Nb}}^{\ce{L}} \partial x_{\ce{Cr}}^{\ce{L}}}$ & $\frac{\partial^2 f_{\ce{L}}}{\partial(x_{\ce{Nb}}^{\ce{L}})^2}$
		\\\hline
		%
		%
		\end{tabular}
		\end{small}
		\end{table}


	\section{Line Compound Approximations}\label{app:linecomp}
		Initial guesses, and replacement values for parallel tangent iterations which fail to converge,
		are made using line compound approximations based on the pure phase regions observed on the Cr--Nb--Ni phase diagram.
		Specifically, the composition of one species is set equal to a constant value,
		and the other two are scaled from their ``real'' values (as opposed to the ``fictitious'' quantities being solved for)
		to satisfy conservation of mass in each point $\left(\sum x_i=1\right)$.
		Replacements for un-converged values are perturbed with random noise of amplitude $\varepsilon=\num{e-5}$.
		
		\begin{align*}
			x_{\ce{Nb}}^{\gamma} &= 0.015\\
			x_{\ce{Cr}}^{\gamma} &= \frac{x_{\ce{Cr}}}{x_{\ce{Cr}} + x_{\ce{Nb}}^{\gamma} + x_{\ce{Ni}}}\\
			                                      &= \frac{x_{\ce{Cr}}}{1 + x_{\ce{Nb}}^{\gamma} - x_{\ce{Nb}}} 
		\end{align*}
		\begin{align*}
			x_{\ce{Ni}}^{\delta} &= 0.75\\
			x_{\ce{Cr}}^{\delta} &= \frac{x_{\ce{Cr}}}{x_{\ce{Cr}} + x_{\ce{Nb}} + x_{\ce{Ni}}^{\delta}}\\
			x_{\ce{Nb}}^{\delta} &= \frac{x_{\ce{Nb}}}{x_{\ce{Cr}} + x_{\ce{Nb}} + x_{\ce{Ni}}^{\delta}}
		\end{align*}
		\begin{align*}
			x_{\ce{Nb}}^{\ce{L}} &= 0.30\\
			x_{\ce{Cr}}^{\ce{L}} &= \frac{x_{\ce{Cr}}}{1 + x_{\ce{Nb}}^{\ce{L}} - x_{\ce{Nb}}}
		\end{align*}

	\begin{thebibliography}{1}
		\bibitem{Du2005} Du, Y.; Liu, S.; Chang, Y. A.; and Yang, Y.
		                 ``A Thermodynamic Modeling of the Cr--Nb--Ni System.''
		                 \emph{Calphad} \textbf{29} (2005) 140--148.
		                 DOI:~10.1016/j.calphad.2005.06.001.

		\bibitem{Karunaratne2005} Karunaratne, M. S. A. and Reed, R. C.
		                          ``Interdiffusion of Niobium and Molybdenum in Nickel between 900 - \SI{1300}{\degreeCelsius}.''
		                          \emph{Defect and Diffusion Forum} \textbf{237-240} (2005) 420--425.
		                          DOI:~10.4028/www.scientific.net/DDF.237-240.420.

		\bibitem{Kim1999} Kim, S. G.; Kim, W. T. and Suzuki, T.
		                  ``Phase-field model for binary alloys.''
		                  \emph{Phys. Rev. E} \textbf{60} (1999) 7186--7197.
		                  DOI:~10.1103/PhysRevE.60.7186.

		\bibitem{Provatas2010} Provatas, N. and Elder, K.
		                       \emph{Phase-Field Methods in Materials Science and Engineering.}
		                       Wiley-VCH: Weinheim, 2010.
		                       ISBN:~978-3-527-40747-7.

		\bibitem{Xu2016} Xu, G. and Liu, Y. and Kang, Z.
		                          ``Atomic Mobilities and Interdiffusivities for fcc Ni-Cr-Nb Alloys.''
		                          \emph{Met. Trans. B} \textbf{47B} (2016) 3126--3131.
		                          DOI:~10.1007/s11663-016-0726-6.

		\bibitem{Zhou2014} Zhou, N.; Lv, D.; Zhang, H.; McAllister, D.; Zhang, F.; Mills, M. and Wang, Y.
		                   ``Computer simulation of phase transformation and plastic deformation in IN718 superalloy: Microstructural evolution during precipitation.''
		                   \emph{Acta Mater.} \textbf{65} (2014) 270--286.
		                   DOI:~10.1016/j.actamat.2013.10.069.
	\end{thebibliography}
\end{document}
